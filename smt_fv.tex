\section{SMT-based Solidity verification}
\label{section:smt}

SMT solvers are powerful tools to prove satisfiability of formulas in different
logics which often have the necessary expressiveness to model software in a
straightforward manner~\cite{Komuravelli13,Alt17,Donaldson11,Beyer11}.

We translate Solidity contracts and their functions into SMT formulas using a
combination of different quantifier-free theories.
%
We shall name the translated formulas the \emph{SMT encoding} of the Solidity
program.
%
The goal of the translation from Solidity to SMT formulas is to verify safety
properties from the Solidity program by performing queries to the SMT solver.

\subsection{SMT encoding}

The SMT encoding is computed during a depth-first traversal of the
abstract syntax tree (AST) of the
Solidity program and thus roughly follows the execution order.
%
For now, each function is analyzed in isolation and thus the 
context regarding the SMT solver (contract storage, local variables, etc.)
is cleared before each function of a contract is visited.
%
There are five types of formulas that are encoded from Solidity inside each function.
%
Three of them, \emph{Control-flow}, \emph{Type constraint} and \emph{Variable assignment}
are simply translated as SMT constraints.
%
The \emph{Branch conditions} are the conditions of the current branch of
execution and thus grow and shrink as we traverse the AST.
%
The last, \emph{Verification Target}, creates a formula consisting of the
verification goal conjoined with the previously mentioned constraints,
including the current branch conditions, and
queries the SMT solver for satisfiability.
%
The different types of encoding are described below.

\begin{paragraph}{Branch conditions.}
For an if-statement \code{if ($c$) $T$ else $F$}, we add
$c$ to the branch conditions during the visit of $T$.
After that, we replace $c$ by $\neg c$
for the visit of $F$ and also remove that when we are
finished with the if-statement.
\end{paragraph}
  
\begin{paragraph}{Control-flow.}
These constraints model conditional termination of execution.
A \code{require($r$)} statement (and similar for \code{assert($r$)})
terminates execution if $r$ evaluates to $\mathrm{false}$, but
of course only if it is executed. Thus,
we add a constraint $b \to r$, where $b$ is the conjunction of
the current branch conditions. Note that due to the implication,
we can keep this constraint even when we leave the current branch.
\end{paragraph}

\begin{paragraph}{Type constraint.}
A variable declaration leads to a correspondent SMT variable that is assigned
the default value of the declared type.
%
For example, Boolean variables are assigned $\mathrm{false}$, and integer variables are
assigned 0.
%
Function parameters are initialized with a range of valid values for the given
type, since their value is unknown.  For instance, a parameter \code{uint32 x}
is initialized as $0 \le x < 2^{32}$ (32 bits), a parameter \code{int256 y}
is initialized as $-2^{255} \le y < 2^{255}$, and a parameter \code{address
a} is assigned the range $0 \le a < 2^{(8*20)}$ (20 bytes).
%
The encoder currently supports Boolean and the various sizes of Integer
variables.

\end{paragraph}

\begin{paragraph}{Variable assignment.}
The encoding of a variable assignment follows the \emph{Single Static Assignment}
(SSA) where each assignment to a program variable introduces a new SMT variable
that is assigned to only once.
%
When a program variable is modified inside different branches of execution,
a new variable is created after the branch to re-combine the different
values after the branches.
We use the if-then-else-function \code{ite} to assign the value
\code{ite(c, $x_1$, $x_2$)} (if-then-else), where $c$ is the branch
condition and $x_1$ and $x_2$ are the two SSA variables corresponding to $x$
at the ends of the branches (cf.\ the $\phi$ function in SSA).
\end{paragraph}

\begin{paragraph}{Verification target.}
Every arithmetic operation is checked against underflow and overflow according
to the type of the values, and an example is given if there is an underflow or
overflow.
%
We also check whether branch conditions are constant, warning the user about
unreachable blocks or trivial conditions.
%
The conditions in calls to \code{assert} represent target postconditions that the
Solidity programmer wants to ensure at runtime and are verified statically.
%
If it is possible to disprove the assertion provided that the control flow
can reach it (i.e.\ the current branch conditions are satisfiable),
the user is given a counterexample.
%
In contrast, \code{require} conditions are meant to be used as filters for
unwanted input values when they are unknown, for example, in public functions,
acting like preconditions for the rest of the scope.
%
Therefore, failing calls to \code{require} are not treated as errors and are
just checked for triviality and reachability.
\end{paragraph}

Figure~\ref{figure:solidity_encoding_1} shows on the left a Solidity sample
that requires all five types of encoding, shown on the right, in order to
verify the intended properties.
%
Since the variables \code{uint256 a} and \code{uint256 b} are function
parameters, they are initialized (lines 1 and 2) with the valid range of values
for their type (\code{uint256}).
%
If \code{a = 0}, the \code{require} condition about \code{b} is used as a
precondition when verifying the assertion in the end of the function (line 3).
%
The next two assignments to \code{b} create the new SSA variables
$b_1$ and $b_2$ (line 4).
%
Variable $b_3$ encodes the second and third conditions, and $b_4$
encodes the first condition (lines 5 and 6).
%
Finally, $b_4$ is used in the assertion check (line 7).
%
Note that the nested control-flow is implicitly encoded in the $ite$
variables $b_3$ and $b_4$.
%
We can see that the target assertion is safe within its function.

\begin{figure}
\label{figure:solidity_encoding_1}
\noindent\begin{minipage}{.48\textwidth}
\begin{verbatim}
contract C
{
  function f(uint256 a, uint256 b)
  {
    if (a == 0)
      require(b <= 100);
    else if (a == 1)
      b = 1000;
    else
      b = 10000;
    assert(b <= 100000);
  }
}
\end{verbatim}
\end{minipage}\hfill
\begin{minipage}{.48\textwidth}
1. $a_0 \ge 0 \land a_0 < 2^{256}  \land \phantom{x}$\\
2. $b_0 \ge 0 \land b_0 < 2^{256}  \land \phantom{x}$\\
3. $(a_0 = 0) \rightarrow (b_0 \le 100) \, \land$\\
4. $b_1 = 1000 \land b_2 = 10000$\\
5. $b_3 = ite(a == 1, b_1, b_2) \land \phantom{x}$\\
6. $b_4 = ite(a == 0, b_0, b_3) \land \phantom{x}$\\
7. $\neg b_4 \le 100000$
\end{minipage}
\caption{SMT encoding of an assertion check.}
\end{figure}

It is important to highlight that errors are irrelevant if they result in a
state change reversion (Sec.~\ref{section:smart_contracts}). The user should be warned
about checks such as overflow only if they do not result in a state reversion.
%
One popular example is the SafeMath~\cite{SafeMath} contract which
is commonly used to turn wrapping arithmetics into overflow-checked arithmetics:
\todo{should we move that to ``future plans''?}

\begin{verbatim}
  function add(uint256 a, uint256 b) internal pure
        returns (uint256) {
    uint256 c = a + b;
    require(c >= a);
    return c;
  }
\end{verbatim}

Although the tool detects an overflow in the computation of \code{a + b},
the overflow will result in a truncation of \code{c} in two's complement and thus
any execution that contains the overflow will revert at the \code{require}.
%
In this case the user should not be warned of the error, since no erroneous cases
exist in accepted executions.

As described above, the component performs several local checks during a single
run, therefore it is critical that the used SMT solver supports
incremental checking.
%
Moreover, we do not abstract difficult operations such as multiplication
between variables, and rather try to give precise answers when possible.
%
Therefore we combine various quantifier-free theories, such as Linear
Arithmetics, Uninterpreted Functions and Nonlinear Arithmetics. 
%
Solidity has integrated Z3~\cite{Z3} and CVC4~\cite{CVC4} via their C++ APIs.
%
The two SMT solvers are used together to increase solving power.
%
This has been important especially for the programs that require Nonlinear
reasoning, since often one solver is able to prove a property that the other
cannot.
%
The component is also able to generate \code{smtlib2}~\cite{SMTLIB}
formulas in order to interface with additional solvers.

\subsection{Specific Examples}

\todo{I'm in general not sure about this section. If we put examples that we
can already verify, there aren't many. If we put examples of things we are not
yet able to verify, shouldn't they be in future plans?}

\todo{Put some specific examples here where the solver already is helpful.}

Even though the current implementation of the SMT module supports a small
subset of Solidity, it can already be used to detect flaws that might be
overlooked by the user.
%
We present now a few examples of buggy code that the compiler is already able
to detect.

The following loop is infinite due to variable \code{i} being unsigned.
%
In that case, the user receives a message about the loop's condition being
always true.
\todo{maybe find more interesting example,
which cannot be detected by looking at local types alone}

\begin{verbatim}
for (uint i = owners.length; i >= 0; i--)
{
    // ...
}
\end{verbatim}

Another type of problem that the compiler already finds automatically is
unreachable code.
%
In the following control-flow expressions, it warns the user that the condition
in the \code{else if} is unreachable.

\begin{verbatim}
if (a >= 7) { ... }
else if (a >= 10) { ... }
\end{verbatim}

\todo{maybe another example, using require/assert}

\subsection{Future plans}

We introduce now the features that we intend to implement in the SMT module, as
well as discuss arising research problems.

Our current implementation plans for the component involve supporting
a larger subset of the language, including more complex data structures
such as \code{mapping}.
%
This is especially important for cases such as token contracts, where
properties such as funds leakage and wrong balance could be used as targets.
%
The component is meant to be built as a Bounded Model Checker, unrolling loops
up to a constant bound and automatically detecting bounds when possible.
%
We also intend to introduce a loop pre and postconditions syntax to help the
unbounded case.

\begin{paragraph}{Range restriction of real life values.}
Some Solidity environment variables have a 256 bit unsigned integer type,
although the range of their values is much more restricted in practice.
For instance, the UNIX timestamp of the current block in seconds,
\texttt{block.timestamp} will not exceed 64 bits for the next
500 billion years. To reduce the false positives rate for
overflows, it makes sense to restrict the value range for these
variables in the SMT encoding. It is an open question how to do
this properly, since a straightforward hard cap at some point
could create undesired artefacts around that point.
%
Another environment variable that could have a similar
behavior is \texttt{block.number}.
\end{paragraph}

\begin{paragraph}{Aliasing.}
In many languages, complex data structures are only assigned by
reference, creating two names for the same object and thus changes
performed via one name also affect references via the second name.
This is of course a big challenge for formal verification and
is known as the \emph{aliasing problem}.
This is also the case for some aspects of Solidity, but data stored
in storage does not have this problem: The structure of
storage is determined at compile-time, and all objects are
statically allocated; while arrays can grow, their position in
storage is fixed at compile-time. Because of that, the aliasing problem
is not an issue, as long as we can assume that there are no hash
collisions in keccak256 and dynamic arrays are small enough.
\end{paragraph}

\begin{paragraph}{Multi-transaction invariants.}
One of the most interesting aspects we intend to research and support is
multi-transaction invariants.
%
The ultimate goal is to compute invariants for state variables (resident in the
contract's storage) considering any arbitrary number of calls to the contract.
%
This would enable these invariants to be used as preconditions whenever they
are accessed.
%
As an example, take contract \code{Token} from Sec.~\ref{section:solidity}.
%
We can see from the constructor that the deployer of the contract receives
10000 tokens.
%
The only way to move tokens is via the function \code{transfer}, which decreases a
certain amount of tokens from one account, if it owns enough, and increases the
same amount in another account.
%
As we can see, the number of total tokens never changes and the invariant
$$\sum_{a \in \mathrm{balances}} \mathrm{balances}[a] = 10000$$ holds in the beginning of any
function of the contract.

\todo{Add another more interesting example, maybe with a diagram}
\end{paragraph}

\begin{paragraph}{Modifiers as pre-postconditions.}
A feature that promises to help the computation of state variable invariants
(discussed above) is pre and postconditions for functions.
%
The first challenge is finding a good syntax to represent the conditions.
%
Modifiers (Sec.\ref{section:solidity}) are a natural candidate for that, given
their ability to behave as patterns that wrap functions.
%
The modifier \code{onlyOwner} introduced in Sec.~\ref{section:solidity} acts
exactly as a precondition for any function that uses it, therefore we can
encode it as the predicate $onlyOwner(f)$ which is conjoined with the encoding
of function \code{f}.
%
With this feature we intend to compute functions pre and postconditions
automatically.
\end{paragraph}

\begin{paragraph}{Effective Callback Freeness.}
The idea of \emph{Effective Callback Freeness} was recently introduced by
\cite{Grossman}.
%
A smart contract $C$ is effectively callback free, if any state change caused
by an internal callback can also be caused by an execution that does not have this callback.
%
The authors show that most of the contracts deployed on Ethereum have this
property.
%
This is a powerful property, since it means that any invariant computed for a
contract's state variables still holds even after calling external contracts
with unknown behavior.
%
We intend to study how to integrate this approach to our static analysis.
\end{paragraph}

